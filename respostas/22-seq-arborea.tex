\begin{lemma} \label{lemma:seqarb:existearvore}
    Para toda sequência arbórea $S$, existe uma árvore cuja sequência de graus é $S$.
\end{lemma}

\enlargethispage{1em}
\begin{proof}
    Vamos provar por indução no tamanho $n$ da sequência. Note que, pela \cref{prop:seqarb:tamanhodois}, $n \geq 2$.

    \begin{ncasos}
        \item[Caso base:] $n = 2$. Seja $S = (s_1, s_2)$ uma sequência arbórea. Logo, $s_1 + s_2 = 2$ e, como $s_1$ e $s_2$ devem ser positivos, temos que $s_1 = s_2 = 1$.

        Considere o grafo simples $T = (\set{v_1, v_2}, \set{v_1 v_2})$. Note que $T$ é conexo e acíclico, portanto, $T$ é uma árvore. Além disso, temos que $\grau(v_1) = \grau(v_2) = 1$, então a sequência de graus de $T$ é $(1, 1) = S$, como proposto.

        \item[Hipótese indutiva:] Suponha um $n \geq 2$ tal que para toda  sequência arbórea $S$ de tamanho $n$ existe um árvore com sequência de graus igual a $S$.

        \item[Passo indutivo:] Seja $S = (s_1, \ldots, s_n, s_{n+1})$ uma sequência arbórea de tamanho $n + 1 \geq 2 + 1 > 2$. Logo, pelo \cref{corol:seqarb:decresce}, temos um índice $k$ tal que $s_k > s_{k + 1}$. Assim, podemos construir a sequência $S' = (s_1, \ldots, s_{k - 1}, s_k - 1, s_{k + 1}, \ldots, s_n)$ sem $s_{n + 1}$ e com $s'_k = s_k - 1$, que também é crescente, pois $s'_k \geq s'_{k + 1}$. Ademais, pela \cref{prop:seqarb:ultimoum}, temos que
        \begin{align*}
            \sum_{i = 1}^n s'_i &= \sum_{i = 1}^n s_i - 1 = \left(\sum_{i = 1}^{n + 1} s_i - s_{n + 1}\right) - 1 \\
                &= \sum_{i = 1}^{n + 1} s_i - 2 = \left(2 (n + 1) - 2\right) - 2 \\
                &= 2n - 2
        \end{align*}

        Logo, $S'$ é uma sequência arbórea e, pela hipótese indutiva, temos uma árvore $T' = (V, E)$ cuja sequência de graus é igual a $S'$. Além disso, temos um vértice $v_k \in V$ tal que $\grau_{T'}(v_k) = s_k$.

        Seja $v_{n + 1} \not\in V$ um novo vértice e considere o grafo $T = (V \cup \set{v_{n + 1}}, E \cup \set{v_k v_{n + 1}})$. Por $v_k$ ser o único novo vértice, $T$ continua simples e acíclica. Além disso, $v_k v_{n + 1}$ conecta o novo vértice em $T'$, fazendo com que $T$ também seja conexo e, portanto, uma árvore.

        Por fim, como $v_k v_{n + 1}$ é a única nova aresta de $T'$,  $\grau_T(v_k) = \grau_{T'}(v_k) + 1 = s'_k + 1 = s_k$ e $\grau_T(v_{n + 1}) = 1 = s_{n + 1}$, mantendo os demais graus. Portanto, $S$ é a sequência de graus de $T$. \qedhere
    \end{ncasos}
\end{proof}
