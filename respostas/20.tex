Seja $d_1 \geq d_2 \geq \cdots \geq d_n$ uma sequência de inteiros positivos. Prove que $d = (d_1, \ldots, d_n)$ é uma sequência de graus de alguma \textbf{árvore} se, e somente se, $\sum_{i=1}^n d_i = 2n - 2$. Por exemplo, existe uma árvore cuja sequência de graus é $(4,3,2,1,1,1,1,1)$, mas não existe uma árvore cuja sequência é $(4,3,3,2,1,1,1,1)$.

\itemdsep

\begin{lemma}
    Seja $T = (V, E)$ uma árvore não-vazia. Então,
    \[
        \sum_{v \in V} \grau_T(v) = 2 \abs{V} - 2
    \]
\end{lemma}


\begin{definition}[Sequência Arbórea] \label{def:seqarb}
    \setlength{\belowdisplayskip}{0pt}
    Uma sequência finita $S = (s_1, s_2, \ldots, s_n)$ de inteiros positivos é \textit{arbórea} se for não-crescente e a soma de seus elementos é $2n - 2$. Isto é,
    \[
        s_1 \geq s_2 \geq \cdots \geq s_n \geq 1
        \qquad \text{ e } \qquad
        \sum_{i = 1}^n s_i = 2 n - 2
    \]
\end{definition}

\begin{proposition}
    Seja $S = (s_1, \ldots, s_n)$ uma sequência arbórea de tamanho $n$. Então:

    \begin{enumerate}[
        label = {\alph*)},
        ref = \thetheorem.\alph*,
        parsep = 0pt,
        itemsep = 0.2em,
        topsep = 0pt
    ]
        \item $n \geq 2$;
        \label[proposition]{prop:seqarb:tamanhodois}

        \item $s_n = 1$;
        \label[proposition]{prop:seqarb:ultimoum}

        \item se $n > 2$, então $s_1 > 1$.
        \label[proposition]{prop:seqarb:primeirogrande}
    \end{enumerate}
\end{proposition}

\begin{proof}[Demonstração do \textref{prop:seqarb:tamanhodois}{item a)}]~

    Por $n$ ser o tamanho, temos que $n \in \natural$. No entanto, se $n = 0$, a soma de $S$ é zero, que não condiz com $2 \cdot n - 2 = -2$ e $S$ não poderia ser arbórea.

    Além disso, se $n = 1$ então a soma é $s_1 = 0$, que não é positivo. Logo, $S$ também não poderia ser arbórea.

    Portanto, temos que $S$ só é possível com $n \geq 2$.
\end{proof}

\begin{proof}[Demonstração da \textref{prop:seqarb:ultimoum}{item b)}]~

    Suponha que $s_n \ne 1$. Então, como $s_n$ deve ser positivo, só resta que $s_n \geq 2$. Por definição, $S$ é não-crescente, isto é, $s_1 \geq \cdots \geq s_n$, portanto
    \[
        \sum_{i = 1}^n s_i \geq \sum_{i = 1}^n s_n \geq \sum_{i = 1}^n 2 = 2 n > 2 n - 2
    \]

    Logo, a soma de $S$ não é $2n - 2$, ou seja, $S$ não é arbórea.

    Como $S$ é arbórea, $s_n$ deve ser igual a 1.
\end{proof}

\begin{proof}[Demonstração da \textref{prop:seqarb:primeirogrande}{item c)}]~

    Suponha que $s_1 = 1$. Como $S$ é não-crescente e só contém inteiros positivos, $1 = s_1 \geq \cdots \geq s_n \geq 1$, ou seja, $s_i = 1$ para todo $1 \leq i \leq n$. Logo,
    \[
        \sum_{i = 1}^n s_i = \sum_{i = 1}^n 1 = n = 2n - 2
    \]

    O que resulta em $n = 2$. Portanto, pela contrapositiva, se $n \ne 2$, temos que $s_1 \ne 1$.

    Então, para um $n > 2$ e como $s_1$ deve ser positivo, segue que $s_1 > 1$.
\end{proof}

\begin{corollary} \label{corol:seqarb:decresce}
    Seja $S = (s_1, \ldots, s_n)$ uma sequência arbórea de tamanho $n > 2$. Então existe um índice $1 \leq i < n$ onde $S$ é estritamente decrescente, isto é,
    \[
        s_1 \geq \cdots \geq s_{i-1} \geq s_i > s_{i+1} \geq \cdots \geq s_n \geq 1
    \]
\end{corollary}

\begin{proof}
    Pelas \cref{prop:seqarb:ultimoum,prop:seqarb:primeirogrande}, temos que $s_1 > s_n = 1$, ou seja, o primeiro elemento é maior que o último. Portanto, pelo menos um elemento deverá ser maior que seu sucessor em $S$.
\end{proof}

\begin{lemma} \label{lemma:seqarb:existearvore}
    Para toda sequência arbórea $S$, existe uma árvore cuja sequência de graus é $S$.
\end{lemma}

\enlargethispage{1em}
\begin{proof}
    Vamos provar por indução que para o teorema vale para toda sequência arbórea de tamanho $n$. Note que, pela \cref{prop:seqarb:tamanhodois}, $n \geq 2$.

    \begin{ncasos}
        \item[Caso base:] $n = 2$. Seja $S = (s_1, s_2)$ uma sequência arbórea. Logo, $s_1 + s_2 = 2 \cdot n - 2 = 2$ e, como $s_1$ e $s_2$ devem ser positivos, temos que $s_1 = s_2 = 1$.

        Considere o grafo simples $T = (\set{v_1, v_2}, \set{e})$, com $v_1$ e $v_2$ quaisquer e $e = v_1 v_2$. Note que $T$ é conexo e acíclico, visto que $e$ conecta todos os vértices de $T$ e $m(T) < 2$, portanto, $T$ é uma árvore.

        Além disso, temos que $\grau_T(v_1) = \grau_T(v_2) = \abs{\set{e}} = 1$, então a sequência de graus de $T$ é $(1, 1) = S$, como proposto.

        \item[Passo indutivo:] Suponha um $n \geq 2$ tal que para toda  sequência arbórea $S$ de tamanho $n$ existe um árvore com sequência de graus igual a $S$. Seja $S = (s_1, \ldots, s_n, s_{n+1})$ uma sequência arbórea de tamanho $n + 1 \geq 2 + 1 > 2$.

        Logo, pelo \cref{corol:seqarb:decresce}, temos um índice $k$ tal que $s_k > s_{k + 1}$ e, pela \cref{prop:seqarb:ultimoum}, $s_{n + 1} = 1$. Assim, considere a sequência $S' = (s_1, \ldots, s_{k - 1}, s_k - 1, s_{k + 1}, \ldots, s_n)$ sem $s_{n + 1}$ e com $s'_k = s_k - 1$. Como $S$ era não-crescente e $s_{k - 1} \geq s_k > s_{k + 1}$, então $s'_{k - 1} > s'_k \geq s'_{k + 1}$, ou seja, $S'$ também é não-crescente. Ademais,
        \begin{align*}
            \sum_{i = 1}^n s'_i &= \sum_{i = 1}^{k - 1} s'_i + s'_k + \sum_{i = k + 1}^n s'_i + (1 + 1 - 2) \\
            &= \sum_{i = 1}^{k - 1} s'_i + \left(s'_k + 1\right) + \sum_{i = k + 1}^n s'_i + 1 - 2 \\
            &= \sum_{i = 1}^{k - 1} s_i + s_k + \sum_{i = k + 1}^n s_i + s_{n + 1} - 2 \\
            &= \sum_{i = 1}^{n + 1} s_i - 2
            = (2 (n + 1) - 2) - 2 \\
            &= 2 n - 2
        \end{align*}

        Logo, $S'$ é uma sequência arbórea e, pela hipótese indutiva, temos uma árvore $T' = (V, E)$ cuja sequência de graus é igual a $S'$. Além disso, temos um vértice $v_i \in V$ tal que $\grau_{T'}(v_i) = s_i$.

        Seja $v_{n + 1} \not\in V$ um novo vértice e $e = v_i v_{n + 1}$. Considere também o grafo $T = (V \cup \set{v_{n + 1}}, E \cup \set{e})$. Como $v_{n + 1}$ é novo e incide apenas em $e$, sendo $v_i \ne v_{n +1}$, temos que $T$ também simples e sem ciclos. Além disso, $T'$ era conexo e $v_{n + 1}$ está conectado a $v_i$ em $T$, mantendo a conexidade. Portanto, $T$ é uma árvore.

        Por fim, como $e$ é a única nova aresta de $T'$, $\grau_T(v_j) = \grau_{T'}(v_k) = s'_k = s_k$ para todo $v_k \in V \setminus \set{v_i, v_{n +1}}$. Ademais, $\grau_T(v_i) = \grau_{T'}(v_i) + 1 = s'_i - 1 = s_i$ e $\grau_T(v_{n + 1}) = 1 = s_{n + 1}$. Portanto, $S$ é a sequência de graus de $T$.
    \end{ncasos}
\end{proof}


\begin{theorem}
    Seja $d_1 \geq d_2 \geq \cdots \geq d_n$ uma sequência de inteiros positivos. Então, $d = (d_1, \ldots, d_n)$ é uma sequência de graus de alguma árvore se, e somente se,
    \[
        \sum_{i=1}^n d_i = 2n - 2
    \]
\end{theorem}

\begin{proof}[Demostração \direto]
    Suponha que exista uma árvore $T = (V, E)$ cuja sequência de graus é $d$. Então, podemos considerar $V = \set{v_1, \ldots, v_n}$ tal que $\grau_T(v_i) = d_i$ para todo $1 \leq i \leq n$. Isso implica que $\abs{V} = n$. Assim, pelo \cref{lemma:arvore:somapar},
    \[
        \sum_{v_i \in V} \grau_T(v_i) = 2 \abs{V} - 2 = 2n - 2
    \]

    Portanto,
    \[
        \sum_{i = 1}^n d_i = \sum_{i = 1}^n \grau_T(v_i) = 2 n - 2
    \]
\end{proof}

\begin{proof}[Demostração \inverso]
    Suponha agora que $\sum_{i=1}^n d_i = 2n - 2$. Como $d$ é não-crescente e contém apenas inteiros positivos, então $d$ é uma \hyperref[def:seqarb]{sequência arbórea}. Portanto, pelo \cref{lemma:seqarb:existearvore}, existe uma árvore cuja sequência de graus é $d$.
\end{proof}
