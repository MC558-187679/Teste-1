\documentclass[a4paper, 14pt]{extarticle}

%% Pacotes Gerais %%
\usepackage[utf8]{inputenc}
\usepackage[T1]{fontenc}
\usepackage[brazilian]{babel}


%%%%%%%%%%%%%%%%%%%%%%%%%%%%%
%% Formatação do Documento %%

\usepackage{geometry}
\geometry{
    margin = 1.5cm,
    noheadfoot = true
}


%%%%%%%%%%%%%%%%%
%% Referências %%
\usepackage{nameref, xcolor, url}

\usepackage{hyperref}
\hypersetup{
    pdftitle  = {MC558 - Teste 1},
    pdfauthor = {Tiago de Paula Alves},
    % bookmarks   = true,
    pdfpagemode = UseOutlines,
    %% Cores de Links %%
    colorlinks = true,
    linkcolor  = blue!30!black,
    urlcolor   = red!30!black,
    citecolor  = blue
}

% 'hyperref' com substituição
\newcommand{\textref}[3][\ref]{{%
    \def\swaptext##1{#3}%
    \hyperref[#2]{\swaptext{#1*{#2}}}%
}}

\newcommand{\equref}[1]{%
    \textref{#1}{(##1)}
}

\renewcommand{\url}[1]{
    \href{#1}{\texttt{#1}}
}


%%%%%%%%%%%%%%%%%%%%%%
%% Opções de Seções %%
\usepackage{titlesec, fancyhdr}

% Formatação de seção e subseção
\titleformat{\section}[runin]
    {\titlerule{}\vspace{1ex}\normalfont\large\bfseries}
    {}{.5em}{}[.]
\titleformat{\subsection}[runin]
    {\normalfont\normalsize\bfseries}
    {}{1em}{}[:]

% Separador das seções
\newcommand{\docline}[1][\pagebreak]{%
    ~ \\
    \noindent\rule{\textwidth}{0.4pt}%
    #1
}
% Separador de itens ou subseções
\newcommand{\itemdsep}{
    \noindent\hfil\rule{0.5\textwidth}{.2pt}\hfil
    \vskip1em
}

% Páginas sem numeração
\pagestyle{empty}


%%%%%%%%%%%%%%%%%%%%%
%% Opções do Título %%
\usepackage{titling}

% Título mais pra cima
\pretitle{%
    \vspace{-6em}%
    \begin{center}%
        \Large%
}
\posttitle{%
    \end{center}%
}
% Reduz separação do autor
\preauthor{%
    \vspace{-1.5em}
    \begin{center}%
        \begin{tabular}[t]{c}
}
\postauthor{%
        \end{tabular}%
    \end{center}%
    \vspace{-1.5em}
}
% Sem data
\predate{}\date{}\postdate{}

% Título e autor
\title{
    {\normalsize  MC558 2020s1} \\
    {\LARGE       Teste 1}
}
\author{
    {\normalsize  Tiago de Paula Alves} \\
    {\small       187679}
}


%%%%%%%%%%%%%%%%
%% Documentos %%

\usepackage[nameinlink,noabbrev,brazilian]{cleveref}
\usepackage{enumitem}


%% Ajusta espaços para o Display Mode %%
\newcommand{\reducemathskip}[1][0.5em]{%
    \setlength{\abovedisplayskip}{1pt}%
    \setlength{\belowdisplayskip}{#1}%
    \setlength{\abovedisplayshortskip}{#1}%
    \setlength{\belowdisplayshortskip}{#1}%
}

%% Ambiente para Teoremas %%
\newtheorem{theorem}{Teorema}
\newtheorem{lemma}[theorem]{Lema}
\numberwithin{theorem}{section}

%% Ambiente para Definições %%
\theoremstyle{definition}
\newtheorem{definition}{Definição}
\numberwithin{definition}{section}

%% Lista de Casos %%
\newlist{casos}{enumerate}{2}
\setlist[casos]{
    wide,
    labelwidth    = {\parindent},
    listparindent = {\parindent},
    parsep        = {\parskip},
    topsep        = {0pt},
    label         = {\textbf{Caso \arabic*}:}
}
% \setlist[casos,2]{label=\textbf{Caso \arabic{casosi}\alph*}:}

%% Casos Nomeados: \item[Caso Base:] %%
\newlist{ncasos}{description}{2}
\setlist[ncasos]{
    wide,
    listparindent = {\parindent},
    parsep        = {\parskip},
    topsep        = {0pt}
}

%% Formatação do Cleveref %%
% \crefformat{definition}{#2definição~#1#3}
% \crefformat{proof}{#2demonstração~#1#3}

\usepackage{amsmath, amssymb, bm, mathtools}
\usepackage{etoolbox, xpatch, xspace}
% \usepackage[mathcal]{euscript}
% \usepackage[scr]{rsfso}
\usepackage{mathptmx, relsize, centernot}


\makeatletter

%% Símbolo QED %%
\renewcommand{\qedsymbol}{\ensuremath{\mathsmaller\blacksquare}}

%% Marcadores de Prova: \direto, \inverso %%
\newcommand{\direto}[1][~]{\ensuremath{(\rightarrow)}#1\xspace}
\newcommand{\inverso}[1][~]{\ensuremath{(\leftarrow)}#1\xspace}

\undef\sum
%% Somatório: \sum_i^j, \bigsum_i^j %%
\DeclareSymbolFont{cmex10}{OMX}{cmex}{m}{n}
\DeclareMathSymbol{\sum@d}{\mathop}{cmex10}{"58}
\DeclareMathSymbol{\sum@t}{\mathop}{cmex10}{"50}
\DeclareMathOperator*{\sum}{\mathchoice{\sum@d}{\sum@t}{\sum@t}{\sum@t}}
\DeclareMathOperator*{\bigsum}{\mathlarger{\mathlarger{\sum@d}}}

%% Operadores de Conjunto: \pow(S), \Dom(S), \Img(S) %%
\DeclareSymbolFont{boondox}{U}{BOONDOX-cal}{m}{n}
\DeclareMathSymbol{\pow}{\mathalpha}{boondox}{"50}
\DeclareMathOperator{\Dom}{Dom}
\DeclareMathOperator{\Img}{Im}

\undef\Phi
%% Novo \Phi %%
\DeclareSymbolFont{cmr10}{OT1}{cmr}{m}{n}
\DeclareSymbolFont{cmmi10}{OML}{cmm}{m}{it}
\DeclareMathSymbol{\Phi}{\mathalpha}{cmr10}{"08}
\DeclareMathSymbol{\varpsi}{\mathalpha}{cmmi10}{"20}

\undef\fam
%% Família de Conjuntos: \fam{S} %%
\DeclareMathAlphabet{\fam}{OMS}{cmsy}{m}{n}

\undef\natural
%% Conjuntos Padrões: R, N, Z, C, Q %%
\DeclareMathOperator{\real}{\mathbb{R}}
\DeclareMathOperator{\natural}{\mathbb{N}}
\DeclareMathOperator{\integer}{\mathbb{Z}}
\DeclareMathOperator{\complex}{\mathbb{C}}
\DeclareMathOperator{\rational}{\mathbb{Q}}

%% Definição de Conjuntos: \set{ _ \mid _ } %%
\newcommand{\set}[1]{%
    \begingroup%
        \def\mid{\;\middle|\;}%
        \left\{#1\right\}
    \endgroup%
}

%% Novos Operadores: \modulo, \symdif, \grau %%
\DeclareMathOperator{\modulo}{~mod~}
\DeclareMathOperator{\symdif}{\mathrel{\triangle}}
\DeclareMathOperator{\grau}{deg}

%% Operadores Delimitados: \abs{\sum_i^j}, x \equiv y \emod{n} %%
\newcommand{\abs}[1]{{\left\lvert\,#1\,\right\rvert}}
\newcommand{\emod}[1]{\ \left(\mathrm{mod}\ #1\right)}

\makeatother


% cleveref depois dos outros pacotes
\usepackage[nameinlink,noabbrev,brazilian]{cleveref}

\begin{document}
    \maketitle
    \thispagestyle{empty}

    \section{1}
    \begingroup
        Seja $G$ um grafo $(X,Y)$-bipartido. Suponha que todo vértice em $X$ tem grau $k > 0$ e todo vértice em $Y$ tem grau $r > 0$. Descreva uma identidade (fórmula) envolvendo apenas $k$, $r$, $X$ e $Y$. Usando isto, responda se existe um tal grafo $(X,Y)$-bipartido com $k = 3$, $r = 7$ e $100.000.642$ vértices.

\itemdsep

\begin{lemma} \label{lemma:bipart:grausiguais}
    Seja $G$ um grafo $(X,Y)$-bipartido. Então,
    \[
        \sum_{v \in X} \grau_G(v) = \sum_{v \in Y} \grau_G(v)
    \]
\end{lemma}

\begin{proof}
    Considere a família de conjuntos $\fam{E}$ composta das arestas incidentes a cada elmento de $X$, isto é:
    \[
        \fam{E} = \set{\set{e \in E(G) \mid x \in \varpsi_G(e)} \mid x \in X}
    \]

    Por definição, nenhuma aresta de $G$ é incidente a dois elementos de $X$. Logo, $\fam{E}$ é disjunta por pares. Ademais, como toda aresta de $G$ é incidente a pelo menos um elemento de $X$, a família $\fam{E}$ cobre $E$, ou seja, $\bigcup \fam{E} = E$. Por conta disso, temos que
    \begin{align*}
        \abs{E} &= \sum_{\fam{E}_x \in \fam{E}} \abs{\fam{E}_x} \\
            &= \sum_{x \in X} \abs{\set{e \mid x \in \varpsi(e)}} \\
            &= \sum_{x \in X} \grau(x)
    \end{align*}

    De forma similar, chegamos a uma identidade equivalente para $Y$.

    Portanto,
    \[
        \sum_{x \in X} \grau(x) = \abs{E} = \sum_{y \in Y} \grau(y)
    \]
\end{proof}

\pagebreak

\begin{theorem} \label{thm:bipart:kxry}
    Seja $G$ um grafo $(X,Y)$-bipartido tal que todo vértice em $X$ tem grau $k > 0$ e todo vértice em $Y$ tem grau $r > 0$. Então,
    \[
        k \cdot \abs{X} = r \cdot \abs{Y}
    \]
\end{theorem}

\begin{proof}
    Partindo do \cref{lemma:bipart:grausiguais}, temos que,
    \begin{align*}
        \sum_{v \in X} \grau(v) &= \sum_{v \in Y} \grau(v) \\
        \sum_{v \in X} k &= \sum_{v \in Y} r \\
        \abs{X} \cdot k &= \abs{Y} \cdot r
    \end{align*}
\end{proof}

\subsection{Resposta} ~

Por definição, $X$ e $Y$ são disjuntos, então
\[
    \abs{X} + \abs{Y} = n(G)
\]

Assim, com o \cref{thm:bipart:kxry}, temos que
\begin{align*}
    n &= \abs{X} + \abs{Y} \\
        &= \abs{X} + \frac{k}{r} \abs{X} \\
        &= \frac{r + k}{r} \cdot \abs{X}
\end{align*}

Ou seja,
\[
    \abs{X} = \frac{r \cdot n}{r + k}
\]

Na situação proposta, teríamos que
\[
    \abs{X} = \frac{7 \times 100.000.642}{7 + 3} = 70.000.449,4 \not\in \natural
\]

Como $\abs{X}$ não é inteiro, podemos afirmar que não pode existir um grafo com esses valores.

    \endgroup
    \docline

    \section{2}
    \begingroup
        Seja $d_1 \geq d_2 \geq \cdots \geq d_n$ uma sequência de inteiros positivos. Prove que $d = (d_1, \ldots, d_n)$ é uma sequência de graus de alguma \textbf{árvore} se, e somente se, $\sum_{i=1}^n d_i = 2n - 2$. Por exemplo, existe uma árvore cuja sequência de graus é $(4,3,2,1,1,1,1,1)$, mas não existe uma árvore cuja sequência é $(4,3,3,2,1,1,1,1)$.

\itemdsep

\begin{definition}[Sequência Arbórea] \label{def:seqarb}
    \setlength{\belowdisplayskip}{0pt}
    Uma sequência finita $S = (s_1, s_2, \ldots, s_n)$ de inteiros positivos é \textit{arbórea} se for não-crescente e a soma de seus elementos é $2n - 2$. Isto é,
    \[
        s_1 \geq s_2 \geq \cdots \geq s_n \geq 1
        \qquad \text{ e } \qquad
        \sum_{i = 1}^n s_i = 2 n - 2
    \]
\end{definition}

\begin{proposition}
    Seja $S = (s_1, \ldots, s_n)$ uma sequência arbórea de tamanho $n$. Então:

    \begin{enumerate}[
        label = {\alph*)},
        ref = \thetheorem.\alph*,
        parsep = 0pt,
        itemsep = 0.2em,
        topsep = 0pt
    ]
        \item $n \geq 2$;
        \label[proposition]{prop:seqarb:tamanhodois}

        \item $s_n = 1$;
        \label[proposition]{prop:seqarb:ultimoum}

        \item se $n > 2$, então $s_1 > 1$.
        \label[proposition]{prop:seqarb:primeirogrande}
    \end{enumerate}
\end{proposition}

\begin{proof}[Demonstração do \textref{prop:seqarb:tamanhodois}{item a)}]~

    Por $n$ ser o tamanho, temos que $n \in \natural$. No entanto, se $n = 0$, a soma de $S$ é zero, que não condiz com $2 \cdot n - 2 = -2$ e $S$ não poderia ser arbórea.

    Além disso, se $n = 1$ então a soma é $s_1 = 0$, que não é positivo. Logo, $S$ também não poderia ser arbórea.

    Portanto, temos que $S$ só é possível com $n \geq 2$.
\end{proof}

\begin{proof}[Demonstração da \textref{prop:seqarb:ultimoum}{item b)}]~

    Suponha que $s_n \ne 1$. Então, como $s_n$ deve ser positivo, só resta que $s_n \geq 2$. Por definição, $S$ é não-crescente, isto é, $s_1 \geq \cdots \geq s_n$, portanto
    \[
        \sum_{i = 1}^n s_i \geq \sum_{i = 1}^n s_n \geq \sum_{i = 1}^n 2 = 2 n > 2 n - 2
    \]

    Logo, a soma de $S$ não é $2n - 2$, ou seja, $S$ não é arbórea.

    Como $S$ é arbórea, $s_n$ deve ser igual a 1.
\end{proof}

\begin{proof}[Demonstração da \textref{prop:seqarb:primeirogrande}{item c)}]~

    Suponha que $s_1 = 1$. Como $S$ é não-crescente e só contém inteiros positivos, $1 = s_1 \geq \cdots \geq s_n \geq 1$, ou seja, $s_i = 1$ para todo $1 \leq i \leq n$. Logo,
    \[
        \sum_{i = 1}^n s_i = \sum_{i = 1}^n 1 = n = 2n - 2
    \]

    O que resulta em $n = 2$. Portanto, pela contrapositiva, se $n \ne 2$, temos que $s_1 \ne 1$.

    Então, para um $n > 2$ e como $s_1$ deve ser positivo, segue que $s_1 > 1$.
\end{proof}

\begin{corollary} \label{corol:seqarb:decresce}
    Seja $S = (s_1, \ldots, s_n)$ uma sequência arbórea de tamanho $n > 2$. Então existe um índice $1 \leq i < n$ onde $S$ é estritamente decrescente, isto é,
    \[
        s_1 \geq \cdots \geq s_{i-1} \geq s_i > s_{i+1} \geq \cdots \geq s_n \geq 1
    \]
\end{corollary}

\begin{proof}
    Pelas \cref{prop:seqarb:ultimoum,prop:seqarb:primeirogrande}, temos que $s_1 > s_n = 1$, ou seja, o primeiro elemento é maior que o último. Portanto, pelo menos um elemento deverá ser maior que seu sucessor em $S$.
\end{proof}

\begin{lemma} \label{lemma:seqarb:existearvore}
    Para toda sequência arbórea $S$, existe uma árvore cuja sequência de graus é $S$.
\end{lemma}

\enlargethispage{1em}
\begin{proof}
    Vamos provar por indução que para o teorema vale para toda sequência arbórea de tamanho $n$. Note que, pela \cref{prop:seqarb:tamanhodois}, $n \geq 2$.

    \begin{ncasos}
        \item[Caso base:] $n = 2$. Seja $S = (s_1, s_2)$ uma sequência arbórea. Logo, $s_1 + s_2 = 2 \cdot n - 2 = 2$ e, como $s_1$ e $s_2$ devem ser positivos, temos que $s_1 = s_2 = 1$.

        Considere o grafo simples $T = (\set{v_1, v_2}, \set{e})$, com $v_1$ e $v_2$ quaisquer e $e = v_1 v_2$. Note que $T$ é conexo e acíclico, visto que $e$ conecta todos os vértices de $T$ e $m(T) < 2$, portanto, $T$ é uma árvore.

        Além disso, temos que $\grau_T(v_1) = \grau_T(v_2) = \abs{\set{e}} = 1$, então a sequência de graus de $T$ é $(1, 1) = S$, como proposto.

        \item[Passo indutivo:] Suponha um $n \geq 2$ tal que para toda  sequência arbórea $S$ de tamanho $n$ existe um árvore com sequência de graus igual a $S$. Seja $S = (s_1, \ldots, s_n, s_{n+1})$ uma sequência arbórea de tamanho $n + 1 \geq 2 + 1 > 2$.

        Logo, pelo \cref{corol:seqarb:decresce}, temos um índice $k$ tal que $s_k > s_{k + 1}$ e, pela \cref{prop:seqarb:ultimoum}, $s_{n + 1} = 1$. Assim, considere a sequência $S' = (s_1, \ldots, s_{k - 1}, s_k - 1, s_{k + 1}, \ldots, s_n)$ sem $s_{n + 1}$ e com $s'_k = s_k - 1$. Como $S$ era não-crescente e $s_{k - 1} \geq s_k > s_{k + 1}$, então $s'_{k - 1} > s'_k \geq s'_{k + 1}$, ou seja, $S'$ também é não-crescente. Ademais,
        \begin{align*}
            \sum_{i = 1}^n s'_i &= \sum_{i = 1}^{k - 1} s'_i + s'_k + \sum_{i = k + 1}^n s'_i + (1 + 1 - 2) \\
            &= \sum_{i = 1}^{k - 1} s'_i + \left(s'_k + 1\right) + \sum_{i = k + 1}^n s'_i + 1 - 2 \\
            &= \sum_{i = 1}^{k - 1} s_i + s_k + \sum_{i = k + 1}^n s_i + s_{n + 1} - 2 \\
            &= \sum_{i = 1}^{n + 1} s_i - 2
            = (2 (n + 1) - 2) - 2 \\
            &= 2 n - 2
        \end{align*}

        Logo, $S'$ é uma sequência arbórea e, pela hipótese indutiva, temos uma árvore $T' = (V, E)$ cuja sequência de graus é igual a $S'$. Além disso, temos um vértice $v_i \in V$ tal que $\grau_{T'}(v_i) = s_i$.

        Seja $v_{n + 1} \not\in V$ um novo vértice e $e = v_i v_{n + 1}$. Considere também o grafo $T = (V \cup \set{v_{n + 1}}, E \cup \set{e})$. Como $v_{n + 1}$ é novo e incide apenas em $e$, sendo $v_i \ne v_{n +1}$, temos que $T$ também simples e sem ciclos. Além disso, $T'$ era conexo e $v_{n + 1}$ está conectado a $v_i$ em $T$, mantendo a conexidade. Portanto, $T$ é uma árvore.

        Por fim, como $e$ é a única nova aresta de $T'$, $\grau_T(v_j) = \grau_{T'}(v_k) = s'_k = s_k$ para todo $v_k \in V \setminus \set{v_i, v_{n +1}}$. Ademais, $\grau_T(v_i) = \grau_{T'}(v_i) + 1 = s'_i - 1 = s_i$ e $\grau_T(v_{n + 1}) = 1 = s_{n + 1}$. Portanto, $S$ é a sequência de graus de $T$.
    \end{ncasos}
\end{proof}

\begin{lemma}
    Seja $T = (V, E)$ uma árvore não-vazia. Então,
    \[
        \sum_{v \in V} \grau_T(v) = 2 \abs{V} - 2
    \]
\end{lemma}


\begin{theorem}
    Seja $d_1 \geq d_2 \geq \cdots \geq d_n$ uma sequência de inteiros positivos. Então, $d = (d_1, \ldots, d_n)$ é uma sequência de graus de alguma árvore se, e somente se,
    \[
        \sum_{i=1}^n d_i = 2n - 2
    \]
\end{theorem}

\begin{proof}[Demonstração {\direto[]}]
    Suponha que exista uma árvore $T = (V, E)$ cuja sequência de graus é $d$. Então, podemos considerar $V = \set{v_1, \ldots, v_n}$ tal que $\grau(v_i) = d_i$ para todo $1 \leq i \leq n$. Isso implica que $\abs{V} = n$ e, portanto,
    \begin{align*}
        \sum_{i = 1}^n d_i &= \sum_{i = 1}^n \grau(v_i) \\
            &= 2 \abs{E} = 2 m(T) \tag*{\equref{thm:slide:sd2e}} \\
            &= 2 \left(n(T) - 1\right) \tag*{\equref{thm:slide:mn1}} \\
            &= 2 n - 2 \qedhere
    \end{align*}
\end{proof}

\begin{proof}[Demonstração {\inverso[]}]
    Suponha agora que $\sum_{i=1}^n d_i = 2n - 2$. Como $d$ é não-crescente e contém apenas inteiros positivos, então $d$ é uma \hyperref[def:seqarb]{sequência arbórea}. Portanto, pelo \cref{lemma:seqarb:existearvore}, existe uma árvore cuja sequência de graus é $d$.
\end{proof}

    \endgroup
    \docline

\end{document}
